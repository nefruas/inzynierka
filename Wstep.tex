\chapter{WSTĘP}



Powstaje wiele hal produkcyjnych, które wykorzystują zintegrowaną wizualizacje umożliwiającą operatorowi nadzorowanie i obsługę jednocześnie kilku układów regulacji procesowej. Takie rozwiązanie zapewnia kontrolę nad poszczególnymi procesami, etapami produkcji oraz zwiększa bezpieczeństwo w przypadku awarii gdy procesy są od siebie zależne. Powstał więc pomysł przygotowania wspólnej wizualizacji zrealizowanie na panelach operatorskich dla sześciu procesów, które zostały stworzone w oparciu o stanowiska laboratoryjne.

W tym zostało zaprojektowanie oraz zrealizowane sześć procesów opartych o stanowiska laboratoryjne wraz z wizualizacją, pracujące na sterownikach typu Simatic S7-300.
\\Procesy zostały zrealizowane za pomocą oprogramowania Simatic STEP 7 V5.5. Oprogramowanie to pozwala na tworzenie aplikacji w kilku językach programowania. Użyte zostały języki LAD i STL z powodu największej ich znajomości przez autorów projektu.

Zrealizowane procesy zawierają następujące zagadnienia:
\\1. Projektowanie aplikacji w środowisku Simatic S7-300\\
\a. Symulacja projektu\\
b. Realizacja projektu na rzeczywistym sterowniku\\
2. Obsługa modułów we/wy cyfrowych oraz analogowych\\
3. Obsługa modułów licznikowych FM350-1\\
4. Sterowanie falownikiem MicroMaster 440 za pomocą sterownika PLC\\
5. Obsługa przetwornika impulsowo - obrotowego\\
6. Integracja systemu bezpieczeństwa samosPRO ze stanowiskiem laboratoryjnym\\
7. Odczyt odległości za pomocą czujników ultradźwiękowych Microsonic pico+25/U\\
8. Regulacja pracy silnika za pomocą regulatora PID wbudowanego w sterownik PLC Simatic S7-300\\
9. Tworzenie wizualizacji na ekran dotykowy HMI za pomocą oprogramowania WinCC Flexible\\


Autorzy pracy pracowali w grupach dwuosobowych nad każdym projektem stanowiska laboratoryjnego. Każdy autor odpowiadał w pełni za dwa stanowiska, a przy dwóch innych pełnił rolę pomocniczą.

Lorem ipsum dolor sit amet, consectetuer adipiscing elit. Donec vitae ipsum ut
libero lacinia sodales. Nunc nec diam quis felis congue aliquet. Proin hendrerit
urna eget mauris. Sed vel mauris nec velit dictum luctus
\cite{diduce:DiduceBugs}.

% ex: set tabstop=4 shiftwidth=4 softtabstop=4 noexpandtab fileformat=unix filetype=tex spelllang=pl,en spell:

